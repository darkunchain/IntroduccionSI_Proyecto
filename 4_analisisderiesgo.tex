\newpage
\pagenumbering{arabic}
\section{\large Análisis y gestión del Riesgo}

\subsection{Matriz DOFA} 

\subsubsection{Debilidades}
\begin{itemize}
    \item Software ERP y POS antiguo o desactualizado.
    \item  del personal interno.
    \item Políticas de Seguridad deficientes
\end{itemize}

\subsubsection{Oportunidades}
\begin{itemize}
\item Capacitación del personal en temáticas de seguridad Información.
\item Migración de sistemas y datos a la nube.
\item Sistemas de respaldo para factura electrónica.
\item Protocolos de borrado seguro.
\item Sistemas de respaldo de CCTV en nube o en ubicaciones físicas diferentes.
\end{itemize}

\subsubsection{Fortalezas}
\begin{itemize}
\item Sistemas de respaldo de energía (UPS, plantas eléctricas).
\item Sistemas de control de incendios.
\item CCTV.
\item Seguridad en el control de accesos (alarmas, personal de seguridad, áreas restringidas)
\end{itemize}

\subsubsection{Amenazas}
\begin{itemize}
\item Ataque cibernético o filtración de datos.
\item Ataques de Ingeniería Social.
\item Aumento de la competencia (Ataque dirigido por parte de la competencia)
\item Robo del equipo de computo.
\item Robo de la documentación física
\item Riesgos medio ambientales (No tener servicio de luz, terremotos etc.)
\item Aumento en los precios de materias primas o escasez de los insumos o productos.
\end{itemize}

\subsection{Priorización de Riesgos} 