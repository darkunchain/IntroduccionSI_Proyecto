\newpage

\begin{table}[h]
\centering
\caption{Clasificación de establecimientos}
\begin{tabular}{>{\bfseries}l p{8cm}}
\toprule
\textbf{Tipo} & \textbf{Descripción} \\
\midrule
Tipo A (Tienda Mixta-Superete) & De formato mixto, caja registradora, baja tecnología y de un espacio que oscila entre los 50 y 100 m\textsuperscript{2}. \\
\addlinespace
Tipo B (Superetes Pequeño) & Son los autoservicios con más de una registradora y que ocupan un espacio entre los 50 y 100 m\textsuperscript{2}. \\
\addlinespace
Tipo C (Superetes medianos) & Autoservicios con más de dos cajas, nivel medio alto de tecnología y un espacio entre 101 y 200 m\textsuperscript{2}. \\
\addlinespace
Tipo D (Superetes Grandes) & Autoservicio, más de tres cajas, nivel medio alto de tecnología que miden entre 201 y 400 m\textsuperscript{2}. \\
\bottomrule
\end{tabular}
\end{table}